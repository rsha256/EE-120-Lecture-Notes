\section{Tuesday, September 6th}
\subsection{Euler's \& Inverse Euler's}
\subsubsection{Euler's}
\[
    e^{i1t} = \cos(t)+i\sin(t)
\]
but the speed can be anything:
\[
    e^{i\omega t} = \cos(\omega t)+i\sin(\omega t)
\]
where $\omega$ has units of $\frac{\text{rad}}{\text{sec}}$. Note that $f$ OTOH has units of $\frac{\text{cycles}}{\text{sec}}=\text{Hz}$. also note that $2\pi$ has units of $\frac{\text{rad}}{\text{cycle}}$.

Note that $\omega > 0 \implies \text{CCW movement}$, and $\omega < 0 \implies \text{CW movement}$

\begin{shaded}
Phasor: a vector that rotates around the unit circle.
\end{shaded}

Question:
\begin{shaded}
Given $e^{i\omega t}$ and $e^{-i\omega t}$, can you get $\cos(\omega t)$?
\end{shaded}
Answer:
\[
    e^{i\omega t} + e^{-i\omega t} = 2\cos(\omega t)
    \implies
    \boxed{\cos(\omega t)=\frac{e^{i\omega t} + e^{-i\omega t}}{2}}
\]

\subsubsection{Odd and Even functions}
\[
    \cos(-t) = \cos(t)
\]
\[
    \sin(-t) = -\sin(t)
\]

\subsubsection{Inverse Euler's}
Question:
\begin{shaded}
Given $e^{i\omega t}$ and $e^{-i\omega t}$, can you get $\sin(\omega t)$?
\end{shaded}
Answer:
\[
    e^{i\omega t} - e^{-i\omega t} = 2i\sin(\omega t)
    \implies
    \boxed{\sin(\omega t)=\frac{e^{i\omega t} - e^{-i\omega t}}{2i}}
\]

\subsection{Periodicity}
You have probably seen periodicity before in simple harmonic motion (without dampening) in a Physics class' dynamics unit.

Here is a nice story otherwise:
\begin{shaded}
This story of Romeo and Juliet is taken from a Cornell Professor.

Romeo proposes to Juliet and she rejects him. The rejection discourages him but later on she decides to give him a chance. But being discouraged, he rejects her, discouraging her, but the proposal now makes him interested. And so the \textbf{cycle} continues.
\end{shaded}
This is an example of \textit{cyclic behavior}.

Let us now define \textbf{CT Periodicity}.
\subsubsection{CT Periodicity}
$x:\mathbb R\mapsto \{\mathbb R \cup \mathbb C\}$ is periodic if $(\forall t\in\mathbb R),\quad x(t+T)=x(t)$ (for some) $\exists T>0 $.

We say $T$ is the fundamental period if $T$ is the smallest possible value.

Associated with the fundamental period is the fundamental frequency which is $\omega = \frac{2\pi}T \frac{\text{rad}}{\text{sec}}$ or $f=\frac1T \text{Hz}$.

If $x(t)=C$ for some constant $C$, then the fundamental period is undefined.

Example:
\begin{shaded}
Question: Find the period of
\[
    x(t)=\cos\left(\frac{2\pi}5 t\right)
\]

Answer: 
$x(t+T)=\cos\left(\frac{2\pi}5 (t+T)\right)=\cos\left(\frac{2\pi}5 t+\frac{2\pi}5 T\right)=\cos\left(\frac{2\pi}5 t\right)$

which implies $\frac{2\pi}5 T = 2\pi k\implies T=5k$ for some $k\in\{1,2,3,\ldots\}$. 

$\min k = 1 \implies T = 5 \text{sec}$.
\end{shaded}

\newpage
\subsubsection{DT Periodicity}
If $N$ is the smallest such integer, we call it the fundamental period of $x$.

Therefore the (fundamental) frequency is $\omega_0=\frac{2\pi}N$.

Example:
\begin{shaded}
Question: Find the (fundamental) period of $x(n)=C, \quad\forall n\in\mathbb Z$?

Answer: 
Looking at the Samples (dotplot -- which is just $C$ for all $n$), we can see that $N= 1$ is the fundamental period if the DT signal is constant with $\omega_0=2\pi\frac{\text{rad}}{\text{sample}}$.
\end{shaded}

Example:
\begin{shaded}
Question: Find the fundamental period and frequency of $x(n)=e^{in}, \quad\forall n\in\mathbb Z$?

Answer:
This must hold $(\forall n\in\mathbb Z),$
\[
    e^{i(n+N)} = e^{in}e^{iN} = e^{in}
\]
for some $N\in\mathbb Z$.

But $N=2\pi k\not\in\mathbb Z$.

Therefore the answer is \textbf{not} $2\pi$ as that is not an integer. Therefore the period is undefined.

We can see this graphically as we see that this jumps around the unit circle, never visiting the same point again.
\end{shaded}

Example:
\begin{shaded}
Question: Find the fundamental period and frequency of $x(n)=e^{i\frac\pi4n}, \quad\forall n\in\mathbb Z$?

Answer:
This must hold $(\forall n\in\mathbb Z),$
\[
    e^{\frac\pi4i(n+N)} = e^{\frac\pi4in}e^{\frac\pi4iN} \stackrel{\text{Want}}= e^{\frac\pi4in}
\]
for some $N\in\mathbb Z$.

which happens when $e^{\frac\pi4iN}=1\implies \frac\pi4N=2\pi k\implies k = 1\implies N = 8$.

Therefore we have (fundamental) frequency $\omega_0=\frac{2\pi}8=\frac\pi4$
\end{shaded}

\subsubsection{Necessary and Sufficient Conditions for DT Periodicity}
Now we shall examine the necessary and Sufficient Conditions for $e^{i\omega n}$ to be periodic in $n$.

\[
    e^{i\omega (n+N)} = e^{i\omega n}
\]
\[
    \cancel{e^{i\omega n}}e^{i\omega N} = \cancel{e^{i\omega n}}
\]
\[
    \boxed{e^{i\omega N} = 1}
\]

Now, noting that $\omega$ must be a rational multiple of $\pi$, we get that:
\[
    \omega N = 2\pi k \implies N=\frac{2\pi k}\omega\stackrel{^\dagger}=\frac{2\pi}{\frac l m \pi}k = \frac{2m}l k
\]
\[
    ^\dagger\omega = \frac{2\pi}N k \implies \frac l m = \frac{2k}N\implies \omega\triangleq\frac l m \pi
\]

\subsubsection{CT Periodicity}
\[
    x(t) = e^{i\omega t}
\]
appears to imply that the sky is the limit for $omega$ (i.e. $\omega\to\infty$).

We note that the oscillations become progressivley faster.

Slowest frequency is $\omega = 0$ (constant signal).

\subsubsection{DT Slowest Frequency}
Slowest frequency is $\omega = 0\frac{\text{rad}}{\text{sample}}$.

\subsubsection{Fastest Frequency for Oscillating DT Signal}
For odd multiples of $\pi$,
\[
    e^{i\pi n} = \cos(\pi n) = (-1)^n
\]

\subsection{Filters}
There are 3 types:
\begin{itemize}
    \item Low-Pass Filter
    \item High-Pass Filter
    \item Band-Pass Filter
\end{itemize}

\subsection{Frequency Response of DT-LTI Systems}
There is a special relationship that exists between the Kronecker Delta.
\[
    \delta(n) \ \texttt{->} \ \boxed{H} \ \texttt{->} \  y=\underbrace{h(n)}_\text{impulse response}
\]

and we have this
\[
    x(n) = e^{i\omega n} \ \texttt{->} \ \boxed{H} \ \texttt{->} \  \underbrace{y(n)}_\text{special form}
\]

Now we can convolve:
\begin{align*}
    y(n) &= (h\ast x)(n)
    \\
    &= \sum_m h(m)e^{i\omega(n-m)}
    \\
    y(n) &= \underbrace{\sum_m h(m) e^{-i\omega m}}_{H(\omega)} e^{i\omega n}
\end{align*}

\[
    e^{i\omega n} \ \texttt{->} \ \boxed{\stackrel H h} \ \texttt{->} \  H(\omega)e^{i\omega n}
\]

This is analogous to eigenvalues: $\mathbf A\vec v=\lambda \vec v$ where $\lambda = H(\omega)\in\mathbb C$, $\vec v = e^{i\omega n}$, and where $\mathbf A=$ the system (the box).

We call $H(\omega)$ the frequency response of the LTI system, obtained by plugging the impulse response into the summation: $$H(\omega)=\sum_{m=-\infty}^\infty h(m) e^{-i\omega m}$$.

This property is known as the Eigenfunction Property of Complex Exponentials with respect to a DT-LTI System.

\subsubsection{Modulation}
\[
  x(n) \ \texttt{->} \ \stackrel{\stackrel{e^{i\omega_1 n}}{\stackrel{|}{\texttt{v}}}}{\boxed{X}} \ \texttt{->} \ y(n) = e^{i\omega_1 n}x(n) = e^{i(\omega_0+\omega_1) n}
\]

\subsubsection{DT LTI Valid Examples}
\[
    \alpha_0e^{i\omega_0 n}+\alpha_1e^{i\omega_1 n} \ \texttt{->} \ \stackrel{\text{LTI}}{\boxed{\ }} \ \texttt{->} \  \alpha_0H(\omega_0)e^{i\omega_0 n}
    +
    \alpha_1H(\omega_1)e^{i\omega_1 n}
\]
If for example, we saw a $+ \alpha_3H(\omega_3)e^{i\omega_3 n}$ term then we would know our system is \textbf{not} LTI.

\subsubsection{Frequency Response of 2-pt mvg avg filter}
We want to find the frequency response for the 2-pt moving average filter:
\[
    x(n) \ \texttt{->} \ \boxed{H} \ \texttt{->} \ y(n) = \frac{x(n)+x(n-1)}2
\]
Note that if $x(n)=(-1)^n=e^{i\pi n}\implies y(n)=0\quad\forall n$.

However if $x(n)=1=e^{i0n}\implies y(n)=1\quad\forall n$.

Sanity check values:
\[
    H(\omega=0) = 1
\]
\[
    H(\omega=\pi) = 0
\]

Finding the frequency repsonse:
\begin{shaded}
Finding the impulse repsonse:

Let $x(n)=\delta(n) \implies h(n)=\frac{\delta(n)+\delta(n-1)}2$

then we note the dotplot is $\frac12$ at $n=0$ and $n=1$ and $0$ everywhere else.
\end{shaded}
Now,
\begin{align*}
    H(\omega) &= \sum_m h(m) e^{-i\omega m}
    \\
    &= h(0) + h(1) e^{-i\omega}
    \\
    H(\omega) &= \frac{1+e^{-i\omega}}2
\end{align*}

