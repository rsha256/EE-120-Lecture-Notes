\section{Thursday, September 29th}
\subsection{CT-LTI Freq Resp (contd)}
Given the RC circuit from last time:

\begin{tikzpicture}
  \def\ang{220}
  \def\a{0.9}
  \def\b{0.8}
%   \draw[->,Icol] ({1.5+\a*cos(\ang)},{1+\b*sin(\ang)}) arc (\ang:-40:{\a} and {\b})
%   ;
  \draw (0,0) to[EMF] (0,2) --++(3,0)
              to[thick R] ++(0,-2) to[thick C] (0,0);
  \fill[black] (1.25,2) circle (0.03);
  \draw[line width=0.6] (1.25,2) --++ (0.48,0);
  \node at (-0.35,0.7) {$-$};
  \node at (-0.35,1.4) {$+$};
  \node[minuscol,scale=0.8] at (1.0,-0.25) {$-Q$};
  \node[pluscol,scale=0.8] at (1.98,-0.25) {$+Q$};
  \node[Ccol, scale=0.8] at (1.5,0.6) {$y_C(t)$};
  \node[Rcol, scale=0.8] at (2.1, 1.3) {$y_R(t)$};
\end{tikzpicture}

Circuit diff eq:
\[
    RC\Dot{y}_C(t) + y_C(t) = x(t)
\]

Let $x(t) = e^{i\omega t}\mapsto H_C(\omega) e^{i\omega t}$.

Therefore:
\[
    H_C(\omega) = \frac{\frac1{RC}}{i\omega + \frac1{RC}}
     = \frac{\frac1{RC}}{i\omega - \left(-\frac1{RC}\right)}
\]

This gives us a Low-Pass Filter with a peak at $\frac{\frac1{RC}}{\frac1{RC}}=1$ at 0 and a cutoff frequency of $\omega_c = -3dB=20\log_{10}\left(\frac1{\sqrt{2}}\right)$.

\begin{align*}
    \angle H(\omega)
    &= \angle \frac1{RC} - \angle \vec r
    \\
    &= \angle \frac1{RC} - \angle (i\omega + \frac1{RC})
    \\ 
    &= 0 - \angle (i\omega + \frac1{RC})
    \\
    &= -\arctan(\frac{\omega}{1/RC})
\end{align*}

Note: The phase of 0 is undefined.

\subsubsection{Impulse Response Revisited}
\[
    h_C(t) = \frac1{RC} e^{-\frac t{RC}} u(t)
\]

\[
    g(t) = \beta e^{-\alpha t} u(t) \iff G(\omega) = \frac{\beta}{i\omega + \alpha}
\]

\[
    h_C(t) = \frac1{RC} e^{-t/{RC}} u(t)
    \iff
    H_C(\omega) = \frac{\overbrace{1/{RC}}^\beta}{i\omega + \underbrace{1/{RC}}_{\alpha}}
\]

\hrulefill

\begin{align*}
    RC\Dot{y}_C(\tau) + y_C(\tau) &= x(\tau)
    \\
    RC \Dot{h}_C(\tau) + {h}_C(\tau)
    &= \delta(\tau)
    &&\text{[Note that this is the Dirac Delta]}
    \\
    RC \Dot{h}_C(\tau) e^{\tau/{RC}} + {h}_C(\tau) e^{\tau/{RC}}
    &= \delta(\tau) e^{\tau/{RC}}
    &&\text{[Mult. both sides by non-0 func $e^{\tau/{RC}}$]}
    \\
    &= \delta(\tau)
    \\
    \underbrace{\Dot{h}_C(\tau) e^{\tau/{RC}} + \frac1{RC} {h}_C(\tau) e^{\tau/{RC}}
    = \frac1{RC} \delta(\tau)}_{\dfrac{\mathrm d}{\mathrm d\tau}[h_C(\tau) e^{\tau/{RC}}]=\frac1{RC}\delta(\tau)}
    &= \frac1{RC}\delta(\tau)
    &&\text{[Product Rule of Differentiation]}
\end{align*}

\begin{shaded}
Note that we prefer integration over differentiation as differentiation is a numerically unstable operation when you have noise which comes hand-and-hand with analog systems.
\end{shaded}

\subsection{Integrator-Adder-Gain Block Diagram}
DAG Block: By FTC, we know to get rid of derivatives, we must integrrate!

\begin{align*}
    RC\Dot{y}_C + y_C &= x
    \\
    RC{y}_C + \int y_C &= \int x
    \\
    {y}_C = - \frac1{RC}\int y_C &+ \frac1{RC}\int x
\end{align*}

\subsection{CT-LTI Systems described by LCCDEs}
\subsection{RC-Ckt}
\subsection{Mass-Spring Damper}
See homework 5 q2.
